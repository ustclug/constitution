\documentclass[a4paper, 12pt, twoside]{ctexart}
\usepackage[margin=1in, top=1.35in, bottom=1.35in]{geometry}
\usepackage{titlesec}
\usepackage{fancyhdr}
\usepackage{booktabs}
\usepackage{float}
\usepackage{tabularx}
\usepackage[hidelinks]{hyperref}
\usepackage{fancyhdr}
\linespread{1.5}

\renewcommand{\thesection}{第\chinese{section} 章}
\titleformat{\section}[block]{\centering\bfseries\Large}{\thesection\hspace{0.2in}}{0pt}{}

\newcommand{\draftstatus}{}

\fancypagestyle{firstpage}
{
	\renewcommand{\headrulewidth}{0pt}
	\fancyhead{} % clear all header fields
	\fancyfoot{} % clear all footer fields
	\fancyfoot[LE,RO]{-- \thepage\ --}
	\fancyfoot[LO,RE]{2024年春季修订\draftstatus}
}

\fancypagestyle{otherpage}
{
	\renewcommand{\headrulewidth}{0.5pt}
	\fancyhead{} % clear all header fields
	\fancyhead[RO,LE]{\textbf{中国科学技术大学学生 Linux 用户协会章程}}
	\fancyfoot{} % clear all footer fields
	\fancyfoot[LE,RO]{-- \thepage\ --}
	\fancyfoot[LO,RE]{2024年春季修订\draftstatus}
}

\newcommand{\centercell}[1]{\multicolumn{1}{c}{#1}}

\newcounter{termcounter}
\setcounter{termcounter}{1}
\newcommand{\term}{\paragraph{第\chinese{termcounter}条}\addtocounter{termcounter}{1}}


\title{\bfseries 中国科学技术大学学生 Linux 用户协会章程}
\author{中国科学技术大学学生 Linux 用户协会}
\date{2024年6月4日\thanks{2024年春季学期修订版本:2024年6月2日\textbf{会员代表大会}通过,2024年6月3日 -- 4日交社团指导老师(\textbf{庄严}老师)和\textbf{校学生社团管理指导委员会}审核通过。终稿排版:2024年 -- 2025年 社长 \textbf{罗嘉宏}。}}

\begin{document}
	\maketitle
	
	\tableofcontents
	
	\pagestyle{otherpage}
	\thispagestyle{firstpage}
	
	\section{总则}
	\term 中国科学技术大学学生 Linux 用户协会(下简称“学生 Linux 用户协会”“协会”或“LUG”),是由中国科大学生中的 Linux 用户和开源软件爱好者自发组建的,接受校党委的领导和校团委、校学生社团管理指导委员会的管理与指导的,有组织有纪律的非营利性学生社团。
	
	\term 学生 Linux 用户协会的宗旨是:
	\begin{enumerate}
		\item 以服务在校师生为根本,以回馈开源社区为愿景,为开源事业的发展作出贡献;
		\item 宣传 GNU 与 Linux 文化,弘扬自由软件精神,推广 Linux 系统的应用;
		\item 培养自由软件社区文化氛围,提供 Linux 爱好者交流学习的平台。
	\end{enumerate}
	
	\section{会员}
	
	\term 凡赞成本章程,对 Linux 或开源软件技术有一定了解并有志于推动社团建设的中国科大在校学生,向协会委员会成员提出申请、提交有效身份信息和联系方式后,即成为协会会员。会员离校或向委员会成员提出退会申请后, 失去会员资格。
	
	社团指导教师是协会的特殊成员,协会委员会应按照学校相关规定联系和申请指导教师。指导教师不能在协会委员会任职,但享有学校相关规定赋予的权利和义务。
	
	\term 协会核心会员是在委员会领导下,参与社团管理工作的协会会员。核心会员必须加入技术部或运营部中的至少一部,工作由该部部长统筹组织。协会委员会成员自动成为核心会员。
	
	\term 协会会员有遵守协会章程、参加协会活动的义务;协会会员享有参加会员代表大会、参与委员会选举的选举权和被选举权,享有对协会工作监督和批评的权利,享有使用会员福利设施的权利,享有自由退会的权利。
	
	除上述权利和义务,协会核心会员还负有参与社团管理工作的义务;享有使用协会核心会员的福利设施的权利。
	
	\section{组织架构与管理制度}
	
	\term 本协会的组织原则是民主集中制。会员代表大会是协会的最高权力机构;委员会在会员代表大会委托下代为行使大会的部分职权;技术部和运营部在委员会的领导下进行协会的管理和运营工作。
	
	\term \textbf{会员代表大会}
	
	凡协会会员均可组织会员召开会员代表大会。召开会员大会前,召集人应当知会委员会和学校社团管理指导委员会负责同志,并在协会邮件列表、官方 QQ 群等处发布公告;召开会员代表大会时,到场会员代表应不少于 20 人,并邀请社团管理指导委员会指派的人员到场监督。不符合上述规定的会员代表大会无效。
	
	会员代表大会每学年至少召开一次。凡参加会员代表大会的会员均自动成为会员代表大会代表。会员代表大会的职责是:
	
	\begin{enumerate}
		\item 审议会员资格;
		\item 选举和罢免委员会成员;
		\item 制定和修改协会章程;
		\item 决定协会的活动方针和任务;
		\item 决定其他重大事项。
	\end{enumerate}
	
	会员代表大会闭会期间,由大会选举产生的委员会代为行使职权。
	
	\term \textbf{委员会}
	
	委员会在会员代表大会的委托下代为行使以下权利:
	
	\begin{enumerate}
		\item 定期组织会员代表大会;
		\item 考核和登记新会员;
		\item 执行本章程和会员代表大会决议;
		\item 决定协会的活动方针和任务;
		\item 决定其他重大事项。
	\end{enumerate}
	
	委员会设会长一名,副会长一至三名,负责协会日常事务。委员会对会员代表大会负责。委员会会长任期一年, 一般不得连任,特殊情况下最多可连任一届。
	
	\term \textbf{委员会下属机构}
	
	委员会下设技术部和运营部。技术部负责维护和开发协会所管理的服务器,定期举办技术讲座培训;运营部负责计划和实施协会日常活动、管理协会资产。技术部、运营部各设部长一名,同时兼任协会委员会会长或副会长职务。
	
	技术部、运营部包含若干执行人员,负责具体的管理事务。委员会有权利根据实际需要从协会会员中选拔技术人员,并有权根据工作表现罢免和调整执行人员。
	
	技术部和运营部包含以下执行人员:
	
	\begin{table}[H]
		\centering
		\begin{tabularx}{\textwidth}{cccX}
			\toprule
			\textbf{职务代号} & \textbf{职务} & \textbf{人数} & \centercell{\textbf{职责}} \\
			\midrule
			CFO & 财务 & 1人 & 负责协会的财务,负责更新 LUG 主页财务页面,并负责图书及其他固定资产的管理。 \\
			CTO & 技术 & 2人 & 负责服务器的维护和开发、技术讲座。领导技术部。 \\
			COO & 运营 & 1--2人 & 负责各种活动的实施和组织。领导运营部。\\
			\bottomrule
		\end{tabularx}
	\end{table}
	
	其中 CTO 设正职一名,副职一名。技术部、运营部部长可根据实际情况安排设置其他执行人员。
	
	\section{财务制度}
	
	\term \textbf{财务政策}
	
	协会财务工作由委员会和运营部财务负责人(CFO)协调和负责,接受会员和学校上级单位的监督和质询。
	
	相关负责人应当根据社团运营情况和学校相关规定,建立独立而透明的会计、审计制度。财务工作应当完全透明,相关负责人有详尽记录财务使用情况并定期公开发表财务报告的义务。
	
	对于社会捐款,捐款人有限制捐款用途的权利,有监督和质询捐款使用情况的权利。
	
	\term \textbf{经费来源}
	
	协会资金来源包括:
	
	\begin{enumerate}
		\item 校党委、团委和社团管理指导委员会等上级部门提供的活动经费;
		\item 社团会员自筹的活动经费;
		\item 符合学校规定的社会捐赠。
	\end{enumerate}
	
	协会不向会员收取任何形式的会费。
	
	\term \textbf{经费运用}
	
	协会的经费和捐款,除上级单位和捐款人特殊要求的,不得用于下列用途之外的其他用途:
	
	\begin{enumerate}
		\item 举办经团委审批的社团活动;
		\item 协会网络服务器的硬件维护;
		\item 购置供全体会员使用的福利设施。
	\end{enumerate}
	
	为方便协会活动经费和捐款的运作,委员会可以指定值得信任的个人或第三方机构代为保管经费。在保护相关人员的个人隐私的前提下,应当公开相关账户(包括银行账户、网上支付平台账户)的信息,以便监管和审查。
	
	\term \textbf{财务报告}
	
	协会委员会和财务负责人每年应至少发布一次财务报告,并按照学校相关规定向上级单位提交财务报告。
	
	协会财务工作必须透明,接受会员和学校上级单位监督及质询。协会财务负责人(CFO)对协会整体财务行为负责。
	
	\section{社团章程的修改}
	
	\term 本章程的修改须经由会员代表大会 2/3 多数表决通过、并报请校学生社团管理指导委员会的审批备案后方可生效。
	
	\section{终止程序及终止后的财产处理}
	
	\term \textbf{终止程序}
	
	协会运营出现严重困难时,委员会可以向会员代表大会提出终止协议并要求社团自行解散。经会员代表大会 6/7 多数表决通过和社团指导教师同意,并由校学生社团管理指导委员会批准执行后,可启动终止程序。终止程序由会员代表大会选举的特别委员会执行。
	
	\term \textbf{终止后的财产处理}
	
	协会终止运作后,应当将未使用的捐款返还捐款人。其余无归属的遗留财产全部上缴校学生社团管理指导委员会。
	
	\section{附则}
	
	\term 本章程解释权和修订权归中国科学技术大学学生 Linux 用户协会会员代表大会所有。
	
	\newpage
	\section{附录:2024年春季学期修订发生变更的章程内容}
	
	\begin{center}
		\textbf{本附录“2024年春季学期修订发生变更的章程内容”为本章程的辅助部分,不属于《中国科学技术大学学生 Linux 用户协会章程》正文。}
	\end{center}
	
	\begin{table}[H]
		\centering
		\begin{tabularx}{\textwidth}{XX}
			\toprule
			\centercell{\textbf{2021年秋季版本(上一生效版本)}} & \centercell{\textbf{2024年春季版本(本次生效版本)}} \\
			\midrule
			\textbf{第七条}\ 凡协会会员均可组织会员召开会员代表大会。召开会员大会前,召集人应当知会委员会和学校社团管理指导委员会负责同志,并在协会邮件列表、官方 QQ 群等处发布公告;召开会员代表大会时,到场会员代表应不少于 20 人,并\textbf{由}社团管理指导委员会指派的人员到场监督。不符合上述规定的会员代表大会无效。 & \textbf{第七条}\ 凡协会会员均可组织会员召开会员代表大会。召开会员大会前,召集人应当知会委员会和学校社团管理指导委员会负责同志,并在协会邮件列表、官方 QQ 群等处发布公告;召开会员代表大会时,到场会员代表应不少于 20 人,并\textbf{邀请}社团管理指导委员会指派的人员到场监督。不符合上述规定的会员代表大会无效。 \\
			\midrule
			\textbf{第九条}\ 技术部、运营部包含若干执行人员,负责具体的管理事务。委员会有权利根据\textbf{实}需要从协会会员中选拔技术人员,并有权根据工作表现罢免和调整执行人员。 & \textbf{第九条}\ 技术部、运营部包含若干执行人员,负责具体的管理事务。委员会有权利根据\textbf{实际}需要从协会会员中选拔技术人员,并有权根据工作表现罢免和调整执行人员。 \\
			\midrule
			\textbf{第九条(CFO)}\ 负责协会的财务,负责更新 LUG 主页财务页面\textbf{。} & \textbf{第九条(CFO)}\ 负责协会的财务,负责更新 LUG 主页财务页面\textbf{,并负责图书及其他固定资产的管理。} \\
			\midrule
			\textbf{第九条(CPO)}\ CPO,资产,1--2人,负责图书及其他固定资产的管理。 & \textbf{第九条(CPO)}\ 已被删除。\\
			\midrule
			\textbf{第十三条}\ 协会委员会和财务负责人每\textbf{学期}应至少\textbf{在协会官方网站}发布一次财务报告,并按照学校相关规定向上级单位提交财务报告。 & \textbf{第十三条}\ 协会委员会和财务负责人每\textbf{年}应至少发布一次财务报告,并按照学校相关规定向上级单位提交财务报告。\\
			\bottomrule
		\end{tabularx}
	\end{table}
\end{document}